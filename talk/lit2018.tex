\documentclass[t, utf8, 10pt]{beamer}
\usepackage[ngerman]{babel}
\usepackage{bera}
\usepackage{fontawesome}
\usepackage{pifont}
\usepackage{color}
\usepackage{listings}
\usepackage{url}
\usepackage{ulem}

\setbeamertemplate{navigation symbols}{}
\setbeamercolor{alerted text}{fg=red!70!black}

\newcommand{\soutthick}[1]{%
  \renewcommand{\ULthickness}{2pt}%
  \sout{#1}%
  \renewcommand{\ULthickness}{.4pt}% Resetting to ulem default
}

\lstdefinestyle{customc}{
  belowcaptionskip=1\baselineskip,
  breaklines=true,
  frame=single,
  xleftmargin=\parindent,
  language=C,
  showstringspaces=false,
  basicstyle=\scriptsize\ttfamily,
  keywordstyle=\bfseries\color{green!40!black},
  commentstyle=\itshape\color{purple!40!black},
  identifierstyle=\color{blue},
  stringstyle=\color{orange},
}

\lstdefinestyle{customfortran}{
  belowcaptionskip=1\baselineskip,
  breaklines=true,
  frame=single,
  xleftmargin=\parindent,
  language=Fortran,
  showstringspaces=false,
  basicstyle=\scriptsize\ttfamily,
  keywordstyle=\bfseries\color{green!40!black},
  commentstyle=\itshape\color{purple!40!black},
  identifierstyle=\color{blue},
  stringstyle=\color{orange},
}

\lstdefinestyle{custompython}{
  belowcaptionskip=1\baselineskip,
  breaklines=true,
  frame=single,
  xleftmargin=\parindent,
  language=Python,
  showstringspaces=false,
  basicstyle=\scriptsize\ttfamily,
  keywordstyle=\bfseries\color{green!40!black},
  commentstyle=\itshape\color{purple!40!black},
  identifierstyle=\color{blue},
  stringstyle=\color{orange},
}

\hypersetup{%
  pdftitle={Python im Unterricht}
  ,pdfauthor={Gert-Ludwig Ingold <gert.ingold@physik.uni-augsburg.de>}
  ,pdfsubject={Linux Infotag 2018, Augsburg, 21.4.2018}
  ,pdfkeywords={Python, teaching, jupyter, nbgrader}
}

\graphicspath{{./img/}}

\begin{document}

\begin{frame}
 \vspace{4truecm}
 \begin{center}
  \structure{\LARGE Python im Unterricht}\\[0.3truecm]
  {\large Gert-Ludwig Ingold}

  \vspace{2truecm}
  \faicon{github}
  \texttt{\normalsize git clone https://github.com/gertingold/lit2018.git}
 \end{center}
\end{frame}

\begin{frame}{Über mich}
 \begin{itemize}
  \item seit 1994 Theoretischer Physiker an der Universität Augsburg
  \item seit 2010 auch Vorlesungen zum Programmieren\\
	\url{gertingold.github.io}
  \item 2015--2017 Erasmus+-Projekt iCSE4school\quad
        \raisebox{-0.25truecm}{\includegraphics[width=0.27\textwidth]{eu_flag_co_funded_pos_[rgb]_right}}\\
	Anwendung von SageMath im gymnasialen Unterricht
        \begin{itemize}
	 \item Uniwersytet Śląski, Kattowitz
	 \item Simula Research Laboratory, Oslo
	 \item Universität Augsburg
	 \item Gymnasien in Chorzów und Warschau
	 \item EDU-RES Chorzów
	\end{itemize}
  \item 2017--2019 Erasmus+-Projekt Jupyter@edu\quad
        \raisebox{-0.25truecm}{\includegraphics[width=0.27\textwidth]{eu_flag_co_funded_pos_[rgb]_right}}\\
	Anwendung von Jupyter Notebooks an Universitäten
        \begin{itemize}
	 \item Uniwersytet Śląski, Kattowitz
	 \item Universität Augsburg
         \item European University Cyprus, Nikosia
	 \item Universidade Portucalense, Porto
	 \item EDU-RES Chorzów
	\end{itemize}
 \end{itemize}
\end{frame}

\begin{frame}{Welche Programmiersprache?}

 \vspace{1truecm}
 \uncover<2->{\includegraphics[width=0.5\textwidth]{python-logo-master-v3-TM}}
 \begin{itemize}
  \item Eignet sich die Programmiersprache für Anfänger?
        \uncover<2->{\textcolor{green!70!black}{\ding{52}}}
  \item Eignet sich die Programmiersprache zur Verwendung
        in anderen Fächern, z.B. im naturwissenschaftlichen
        Unterricht?
        \uncover<2->{\textcolor{green!70!black}{\ding{52}}}
  \item Eignet sich die Programmiersprache auch für reale
        Anwendungen in Wissenschaft und Industrie?
        \uncover<2->{\textcolor{green!70!black}{\ding{52}}}
 \end{itemize}

 \vspace{0.5truecm}
 \begin{center}
  \begin{Large}
   \only<3>{Python 2 oder Python 3?}%
   \only<4>{\soutthick{Python 2 oder} Python 3!}%
  \end{Large}
 \end{center}
\end{frame}

\begin{frame}{Was andere denken}
 \begin{columns}
  \begin{column}{0.58\textwidth}
   \includegraphics[width=\textwidth]{Top39-700_4}
  \end{column}
  \begin{column}{0.42\textwidth}
   \vspace{-2truecm}

   \fontsize{5}{6}\selectfont
   http://cacm.acm.org/blogs/blog-cacm/176450-python-is-now-the-most-popular-introductory-teaching-language-at-top-us-universities/fulltext
  \end{column}
 \end{columns}

 \vspace{\baselineskip}
 most popular languages on Github 2017 (\url{octoverse.github.com})

 \includegraphics[width=\textwidth]{Screenshot-2018-4-12_GitHub_Octoverse_2017}
\end{frame}

\begin{frame}[c]{Warum Python?}
 \begin{itemize}
  \item Eigenschaften
        \only<2>{%
        \begin{itemize}
         \item niedrige Lernbarriere durch expressiven und gut lesbaren Code
         \item keine Fixierung auf ein Programmierparadigma
               \begin{itemize}
                \item objektorientiertes Programmieren
                \item Elemente der funktionalen Programmierung
               \end{itemize}
         \item interpretierte Sprache\\
               \begin{itemize}
                \item schnelles Feedback
                \item exploratives Lernen möglich
               \end{itemize}
        \end{itemize}}
  \item Verfügbarkeit
  \item Umfeld
 \end{itemize}
\end{frame}

\begin{frame}[fragile]{Warum Python?}
 \structure{niedrige Lernbarriere durch expressiven und gut lesbaren Code}

 \vspace{0.3truecm}
 \begin{columns}[t]
  \begin{column}{0.5\textwidth}
   Python
   \lstset{style=custompython}
   \begin{lstlisting}
   for n in range(5):
       print(n, n**2)
   \end{lstlisting}

   \begin{footnotesize}
    \begin{itemize}
     \setlength{\itemindent}{-10pt}
     \item Einrückungen sind syntaktisch relevant
     \item keine Typdeklaration (\textit{duck typing})
    \end{itemize}
   \end{footnotesize}
  \end{column}
  \begin{column}{0.5\textwidth}
   Fortran
   \lstset{style=customfortran}
   \begin{lstlisting}
     PROGRAM Squares
     DO n = 0, 4
        PRINT '(2I4)', n, n**2
     END DO
     END PROGRAM Squares
   \end{lstlisting}
  \end{column}
 \end{columns}
  
 \vspace{0.5truecm}
 \begin{columns}
  \begin{column}{0.7\textwidth}
   C
   \lstset{style=customc}
   \begin{lstlisting}
    #include <stdio.h>

    void main(){
       int n;
       for(n = 0; n < 5; n++){
             printf("%4i %4i\n", n, n*n);
       }
    }
   \end{lstlisting}
  \end{column}
  \begin{column}{0.3\textwidth}
   \strut
  \end{column}
 \end{columns}
\end{frame}

\begin{frame}[c]{Warum Python?}
 \begin{itemize}
  \item Eigenschaften
  \item Verfügbarkeit
        \only<2>{%
        \begin{itemize}
         \item frei verfügbar für Windows, MacOS, Linux, \ldots
         \item niedrige Lernbarriere durch expressiven und gut lesbaren Code
         \item umfangreiche Distributionen frei verfügbar\\
               z.B. Anaconda (\url{www.anaconda.com}), \textasciitilde 3\,GB\\
               vor allem für wissenschaftliche Anwendungen, Datenanalyse \ldots
         \item Python kann webbasiert über Jupyterhub zur Verfügung gestellt
               werden
        \end{itemize}}
  \item Umfeld
        \only<3>{%
        \begin{itemize}
         \item umfangreiche freie Programmbibliotheken, auch
               im wissenschaftlichen Bereich
         \item Jupyter Notebook
               \begin{itemize}
                \item webbasiertes Arbeiten mit Python
                \item Möglichkeit, Lernmaterialien zu verteilen
                \item Möglichkeit, die Arbeit im Unterricht zu dokumentieren
               \end{itemize}
         \item nbgrader\\
               System zur Verteilung, dem Einsammeln und Korrigieren von
               Jupyter Notebooks
        \end{itemize}}
 \end{itemize}
\end{frame}

\begin{frame}{Das wissenschaftliche Ökosystem von Python}
 \begin{columns}
  \begin{column}{0.44\textwidth}
   \raisebox{-0.63\textheight}{\includegraphics[width=\textwidth]{sln_logo}}

   \begin{footnotesize}
    \url{https://www.scipy-lectures.org/}
   \end{footnotesize}
  \end{column}%
  \begin{column}{0.56\textwidth}
   \begin{small}
    \begin{itemize}
     \setlength{\itemindent}{-10pt}
     \item Matrizen als Objekte (\texttt{ndarray})
     \item numerische Integration und Lösung von Differentialgleichungen
     \item lineare Algebra
     \item statistische Funktionen
     \item spezielle Funktionen
     \item graphische Darstellungen
     \item Bildbearbeitung
     \item maschinelles Lernen
     \item Bearbeitung großer Datensätze
     \item symbolische Mathematik
     \item und vieles mehr \ldots
    \end{itemize}
   \end{small}
  \end{column}
 \end{columns}
\end{frame}

\begin{frame}{Jupyter notebook}
 \begin{columns}
  \begin{column}{0.6\textwidth}
   \only<1>{\includegraphics[height=0.9\textheight]{notebook_taylor}}%
   \only<2>{\includegraphics[height=0.9\textheight]{notebook_taylor_text}}%
   \only<3>{\includegraphics[height=0.9\textheight]{notebook_taylor_code}}%
   \only<4>{\includegraphics[height=0.9\textheight]{notebook_taylor_math}}%
   \only<5>{\includegraphics[height=0.9\textheight]{notebook_taylor_graph}}%
   \only<6>{\includegraphics[height=0.9\textheight]{notebook_taylor_widget}}%
   \only<7->{\includegraphics[height=0.9\textheight]{notebook_taylor}}%
  \end{column}%
  \begin{column}{0.4\textwidth}
   \vspace{-8truecm}
   \begin{itemize}
    \item \small läuft im Browser
    \item \small Notebookelemente:
          \begin{itemize}
           \item<alert@2> Textzellen
           \item<alert@3> Codezellen
           \item<alert@4> Formeln
           \item<alert@5> graphische Ausgabe
           \item<alert@6> Widgets
          \end{itemize}
    \item \small Umwandlung in verschiedene Formate (HTML, PDF, \ldots) möglich
   \end{itemize}

   \vspace{1truecm}
   \begin{center}
    \begin{Large}
        \uncover<8>{\alert{\fbox{\ding{220} Demo}}}
    \end{Large}
   \end{center}
  \end{column}
 \end{columns}
\end{frame}

\end{document}
