\documentclass[t, utf8, 10pt]{beamer}
\usepackage[ngerman]{babel}
\usepackage{bera}
\usepackage{fontawesome}
\usepackage{listings}
\usepackage{url}

\setbeamertemplate{navigation symbols}{}

\hypersetup{%
  pdftitle={Python im Unterricht}
  ,pdfauthor={Gert-Ludwig Ingold <gert.ingold@physik.uni-augsburg.de>}
  ,pdfsubject={Linux Infotag 2018, Augsburg, 21.4.2018}
  ,pdfkeywords={Python, teaching, jupyter, nbgrader}
}

\graphicspath{{./img/}}

\begin{document}

\begin{frame}
 \vspace{4truecm}
 \begin{center}
  \structure{\LARGE Python im Unterricht}\\[0.3truecm]
  {\large Gert-Ludwig Ingold}

  \vspace{2truecm}
  \faicon{github}
  \texttt{\normalsize git clone https://github.com/gertingold/lit2018.git}
 \end{center}
\end{frame}

\begin{frame}{Über mich}
 \begin{itemize}
  \item seit 1994 Theoretischer Physiker an der Universität Augsburg
  \item seit 2010 auch Vorlesungen zum Programmieren\\
	\url{gertingold.github.io}
  \item 2015--2017 Erasmus+-Projekt iCSE4school\quad
        \raisebox{-0.25truecm}{\includegraphics[width=0.27\textwidth]{eu_flag_co_funded_pos_[rgb]_right}}\\
	Anwendung von SageMath im gymnasialen Unterricht
        \begin{itemize}
	 \item Uniwersytet Śląski, Kattowitz
	 \item Simula Research Laboratory, Oslo
	 \item Universität Augsburg
	 \item Gymasien in Chorzów und Warschau
	 \item EDU-RES Chorzów
	\end{itemize}
  \item 2017--2019 Erasmus+-Projekt Jupyter@edu\quad
        \raisebox{-0.25truecm}{\includegraphics[width=0.27\textwidth]{eu_flag_co_funded_pos_[rgb]_right}}\\
	Anwendung von Jupyter Notebooks an Universitäten
        \begin{itemize}
	 \item Universität Katowice
	 \item Universität Augsburg
         \item European University Cyprus, Nikosia
	 \item Universidade Portucalense, Porto
	 \item EDU-RES Chorzów
	\end{itemize}
 \end{itemize}
\end{frame}

\begin{frame}{Warum Python?}
\end{frame}

\end{document}
